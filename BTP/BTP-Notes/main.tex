\documentclass[a4paper]{article} 
\addtolength{\hoffset}{-2.25cm}
\addtolength{\textwidth}{4.5cm}
\addtolength{\voffset}{-3.25cm}
\addtolength{\textheight}{5cm}
\setlength{\parskip}{0pt}
\setlength{\parindent}{0in}

%----------------------------------------------------------------------------------------
%	PACKAGES AND OTHER DOCUMENT CONFIGURATIONS
%----------------------------------------------------------------------------------------
\usepackage{times}
\usepackage{epsf}
\usepackage{threeparttable}
\usepackage{setspace}
\usepackage{amsmath}
\usepackage{amsthm}
\usepackage{epsfig}
\usepackage{caption}
\usepackage{subfig}
\usepackage{algorithmic}
\usepackage{algorithm}
\usepackage{float}
\usepackage{blindtext} % Package to generate dummy text
\usepackage{charter} % Use the Charter font
\usepackage[utf8]{inputenc} % Use UTF-8 encoding
\usepackage{microtype} % Slightly tweak font spacing for aesthetics
\usepackage[english, ngerman]{babel} % Language hyphenation and typographical rules
\usepackage{amsthm, amsmath, amssymb} % Mathematical typesetting
\usepackage{float} % Improved interface for floating objects
\usepackage[final, colorlinks = true, 
            linkcolor = black, 
            citecolor = black]{hyperref} % For hyperlinks in the PDF
\usepackage{graphicx, multicol} % Enhanced support for graphics
\usepackage{xcolor} % Driver-independent color extensions
\usepackage{marvosym, wasysym} % More symbols
\usepackage{rotating} % Rotation tools
\usepackage{censor} % Facilities for controlling restricted text
\usepackage{listings, style/lstlisting} % Environment for non-formatted code, !uses style file!
\usepackage{pseudocode} % Environment for specifying algorithms in a natural way
\usepackage{style/avm} % Environment for f-structures, !uses style file!
\usepackage{booktabs} % Enhances quality of tables
\usepackage{tikz-qtree} % Easy tree drawing tool
\tikzset{every tree node/.style={align=center,anchor=north},
         level distance=2cm} % Configuration for q-trees
\usepackage{style/btree} % Configuration for b-trees and b+-trees, !uses style file!
\usepackage[backend=biber,style=numeric,
            sorting=nyt]{biblatex} % Complete reimplementation of bibliographic facilities
\addbibresource{ecl.bib}
\usepackage{csquotes} % Context sensitive quotation facilities
\usepackage[yyyymmdd]{datetime} % Uses YEAR-MONTH-DAY format for dates
\renewcommand{\dateseparator}{-} % Sets dateseparator to '-'
\usepackage{fancyhdr} % Headers and footers
\pagestyle{fancy} % All pages have headers and footers
\fancyhead{}\renewcommand{\headrulewidth}{0pt} % Blank out the default header
\fancyfoot[L]{} % Custom footer text
\fancyfoot[C]{} % Custom footer text
\fancyfoot[R]{\thepage} % Custom footer text
\newcommand{\note}[1]{\marginpar{\scriptsize \textcolor{red}{#1}}} % Enables comments in red on marginpar


%----------------------------------------------------------------------------------------

\begin{document}

%-------------------------------
%	TITLE SECTION
%-------------------------------

\fancyhead[C]{}
\hrule \medskip % Upper rule
\begin{minipage}{0.295\textwidth} 
\raggedright
\footnotesize
Subhash \hfill\\   
cs17b005 \hfill\\

\end{minipage}
\begin{minipage}{0.4\textwidth} 
\centering 
\large 
Project Thesis Extra/Continuation\\ 
\normalsize 
Encoder Selection\\ 
\end{minipage}
\begin{minipage}{0.295\textwidth} 
\raggedleft
\today\hfill\\
\end{minipage}
\medskip\hrule 
\bigskip

%-------------------------------
%	CONTENTS
%-------------------------------

\section{Regret Analysis}

\subsection{Martingale Sequence}
A sequence of random variables is called Martingale if the conditional expectation of next vatiable w.r.t all the previous variables is equal to present variable 
$$ E(X_{t+1}|X_t,...,X_2,X_1) = X_t $$
General expectation must also be bounded 
$$ E(X_i) \leq C $$

\subsection{Azuma Hoefding Inequality}
If $Y_t$ is Martingale and diffrence between consecutive terms is bounded i.e. $|Y_t - Y_{t-1}| \leq c$ then we can say that
\begin{equation}
    P{\left[Y_t > c\sqrt{2t\log{\frac{1}{\gamma}}}\right]} \leq \gamma
\end{equation}
This is simpler version of the original inequality and can also be viewed as - 
$ Y_t \leq c\sqrt{2t\log{\frac{1}{\gamma}}} $ holds with probability $ 1-\gamma $

\subsection{Continuation}
We can write the final equation of Regret Analysis as -
$$ reg_t \leq \frac{a \cdot \beta_t}{b + c\cdot\beta_t } $$
Consider $ Y_t = \sum_{s = 1}^t reg_s =  \sum_{s = 1}^t (R\cdot (x_* - x_t)) $ as Martingale. This makes sense because in any bandit problem involving building we construct confidence sets using all the previous variables and pick next action i.e. $ x_t $ is a function of $ x_1, \mu_1, ... , x_{t-1}, \mu_{t-1} $. To strengthen this statment few bandit papers just solve for regret assuming martingale and in Improved Algorithms paper he says a similar term to be martingale. We also know that - 
$$ |Y_t - Y_{t-1}| = reg_t \leq \frac{a \cdot \beta_t}{b + c\cdot\beta_t } \leq \frac{a \cdot \beta_T}{b + c\cdot\beta_T } $$
By using above described inequality with confidence $\delta$
$$ P{\left[Y_T > \frac{a \cdot \beta_T}{b + c\cdot\beta_T} \cdot \sqrt{2T\log{\frac{1}{\delta}}}\right]} \leq \delta $$
So we can say with confidence $1-2\delta$ - As we have used $\delta$ confidence already to get bound on $reg_t$
$$ R(T) = Y_T <= \frac{a \cdot \beta_T}{b + c\cdot\beta_T} \cdot \sqrt{2T\log{\frac{1}{\delta}}} $$






\section{Refered Papers}
\begin{enumerate}
    \item Aldo Pacchiano, Stochastic Bandits with Linear Constraints \\ \url{https://arxiv.org/pdf/2006.10185.pdf}
    \item Yasin Abbasi-Yadkor, Improved Algorithms for Linear Stochastic Banditsm \\ \url{https://papers.nips.cc/paper/2011/file/e1d5be1c7f2f456670de3d53c7b54f4a-Paper.pdf}
\end{enumerate}

%-------------------------------
%	Exaample
%-------------------------------

% \section{First Exercise}
% \blindtext
% \subsection{First Subtask}

% \subsection{Second Subtask}
% \blindtext

% \bigskip

% \section{Second Exercise}
% \blindtext
% \subsection{First Subtask}

% \bigskip

%------------------------------------------------

\end{document}
