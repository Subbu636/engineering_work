\documentclass[10pt]{article}
% Om Sai Ram
\usepackage{palatino,fullpage,url,syntonly,twocolumn}
\usepackage[margin=0.75in]{geometry}
%\syntaxonly
\title{CS 3100: Paradigms of Programming}
\author{Jul.-Nov Semester 2019 \\
'C'  Slot; CS 34 \\
\textbf{Slots are: Mon (10 -- 10.50am); Tue (9 -- 9.50am); Wed (8 -- 8.50am); Fri (1 -- 1.50pm)}; \\
Dr. KC Sivaramakrishnan, BSB 371; Phone: 4359 \\
Email:~\textit{kcsrk@iitm.ac.in} \\
TA(s):
  Shashank Shekhar Dubey (shashank.shekhardubey@gmail.com), \\
	Diptanshu Kakwani (dipkakwani@gmail.com), \\
	Sumit Padhiyar (sumitpad@cse.iitm.ac.in), \\
	Atul Dhiman (cs18m014@smail.iitm.ac.in), \\
	M Prasanna Kumar (CS15B050@smail.iitm.ac.in), \\
	Bikash Behera (CS19M019@smail.iitm.ac.in)
}
\date{Updated on \today}
\begin{document}
\maketitle

Note: Course related communications will be on IITM Moodle site (CS3100);
please regularly check the email that is linked to your email account.

\section{Course objectives}

The aim of the course is to teach you about the different paradigms of
programming, their underlying concepts, and the relationships between them. We
will mostly look at functional, logic and concurrent programming, touching upon
imperative programming concepts as we go through the course. We will study
functional and concurrent paradigm through OCaml and logic paradigm through
Prolog (specifically SWI-Prolog).

\section{Learning Outcomes}

\begin{itemize}

\item To learn about functional programming paradigm through the study of OCaml
	programming language.

\item To learn about logic programming paradigm through the study of Prolog
	programming language.

\item To learn about concurrent programming paradigm through concurrency
	extensions of OCaml.

\item Identify what language features are fundamental and what features are
	syntactic sugar.

\item To be able to write clear, correct, robust, reusable code.

\item To be able to identify the best paradigm for the given problem.

\end{itemize}

\section{Course prerequisite(s)}

CS2200, CS2600, CS2700, CS2710 -- or equivalent for each course.

\section{Classroom Mode}

Traditional Lectures.

\section{Textbooks}

\begin{description}

\item[cs3110] ``Functional Programming in OCaml'', Spring 2019 Edition, CS 3110
	Cornell University, \url{http://www.cs.cornell.edu/courses/cs3110/2019sp/textbook/}.

\item [PLCC] Ravi Sethi, ``Programming Languages : Concepts and Constructs'',
	2nd edition, Addison-Wesley.

\end{description}

\section{Reference Books}

Material from the following books will be used as necessary.

\begin{description}

\item [RWO] Anil Madhavapeddy, Yaron Minsky, Jason Hickey, ``Real World
	OCaml'', O'Reilly Publishers, \url{https://dev.realworldocaml.org/}.

\item [TAPL] Benjamin C. Pierce, ``Types and Programming Languages'', The MIT Press.

\end{description}

\section{Course Requirements}

You are {\em required} to attend all the lectures.  If you miss any of them it
is your responsibility to find out what went on during the classes and to
collect any materials that may be handed out.

Class participation is strongly encouraged to demonstrate an appropriate level
of understanding of the material being discussed in the class.  Regular
feedback from the class regarding the lectures will be very much appreciated.

\section{Planned Syllabus}

The following topics will be covered, but not necessarily in the order
listed below:

\begin{enumerate}
\item Introduction to Programming Languages.
\item Functions, Pattern Matching, Algebraic Data Types
\item Lists, Trees, Sets, Hash Tables, Graphs
\item Lambda Calculus: $\alpha$ equivalence, $\beta$ reduction, $\eta$ expansion.
\item Lambda Calculus: Church encoding, Y-combinator
\item Simply Typed Lambda Calculus, Progress and Preservation
\item Modular Programming in OCaml
\item Monads and Applicative Functors, Generalised Abstract Data Types, Row
	Polymorphism
\item Mutability, Streams and Laziness, Promises, Memoization
\item Logic Programmming, Horn Clauses, Semantics
\item Prolog examples, Arithmetic, Backtracking
\item Generate and Test, Symbolic Execution
\item Lists, Difference Lists, Graph Search
\item Negation as Failure, Cut
\item Databases, Constraint Logic Programming, Databases
\item Continuations, Delimited Continuations, Effect Handlers
\item MVars, Locks and Condition Variables, Selective Communication
\item Asynchronous I/O, Stream Programming, Transactional Memory
\end{enumerate}

\section{Tentative Grading Policy}

The following allocation of points is tentative. These may change
during the semester.

\begin{center}
  \begin{tabular}{ll}
Quiz 1 (Date: Sep. 4, 2019): & 15\% \\
Quiz 2 (Date: Oct. 16, 2019): & 15\% \\
6 Assignments (written / OCaml / Prolog): & 30\% \\
Final Exam (Date: Nov 19, 2019): & 40\%
  \end{tabular}
\end{center}

\section{Academic Honesty}

Academic honesty is expected from each student participating in the course.  NO
sharing (willing, unwilling, knowing, unknowing) of assignment code between
students, submission of downloaded code (from the Internet, Campus LAN, or
anywhere else) is allowed.

Academic violations will be handled by IITM Senate Discipline and Welfare
(DISCO) Committee.  Typically, the first violation instance will result in ZERO
marks for the corresponding component of the Course Grade and a drop of one-
penalty in overall course grade. The second instance of code copying will
result in a 'U' Course Grade and/or other penalties. The DISCO Committee can
also impose additional penalties.

Please protect your Moodle account password. Do not share it with ANYONE. Do
not share your academic disk drive space on the Campus LAN.

\end{document}
